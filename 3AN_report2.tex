\documentclass{article}
\DeclareMathSizes{10}{10}{7}{7}
\usepackage{amsmath}
\usepackage{ amssymb }
\usepackage{tikz, graphicx}
\usepackage{geometry}
\usepackage[makeroom]{cancel}
\DeclareMathOperator{\sech}{sech}
% http://www.mathworks.com/matlabcentral/fileexchange/8015-m-code-latex-package
\usepackage[framed,numbered,autolinebreaks,useliterate]{mcode}
\usepackage{hyperref}
\hypersetup{
    colorlinks=true,
    linkcolor=blue,
    filecolor=magenta,      
    urlcolor=blue,
    }
\usepackage{float}
\restylefloat{table}

\geometry{legalpaper, margin=0.7in}
\title{3AN Project 2\\ What happened when two waves crossed the road}
\author{Liam Watson WTSLIA001}
\begin{document}
\maketitle
\section{Introduction and physics}
We wish to better understand the interaction between colliding solitons within some optical communication media such as firbre obtic cable used for audio or network communications. 
We can model optical pulses in a media with saturation nonlinearity properties using the following modified nonlinear  Schr¨odinger equation:
\begin{align}
i\psi_t + \psi_{xx} + \frac{2|\psi|^2\psi}{1+S\sin(|\psi|^2)} = 0
\end{align}
Note that we impose periodic boundary conditions to the Schr¨odinger equation. \\
We can study the collisions of N solitons by choosing our initial configuration as the sum of N solitons separated by a sufficiently large distance as follows:
\begin{align}
\psi (x,0) = \sum_{j=1}^N A_j \sech(A_j (x-x_j))e^{iv_j(x-x_j)}
\end{align}
In the following sections we will study the soliton collisoons for different values of the nonlinearity saturation $S\in(-1,1)$, the velocity $v$ the intersoliton distance $|x_k-x_j|$. \\
This analysis will be done using two different numerical methods, namely the split-step and finite difference methods. To ensure that any insight gained from either method is consistent we will ensure that the following conserved quantity remains within a maximum deviation of $\varepsilon = 10^-5$.
\begin{align}
N = \int_{-\infty}^\infty |\psi|^2 dx
\end{align}
Additionally we will compare the execution time of both numberical methods for the given $\varepsilon$
\section{Derivation of needed numerical method formula}
DERV
\subsection{Split-Step}
SPLIT Step DERV
\subsection{Finite Difference}
FD derivation
\section{Implementation}
IMPL
\section{Split-Step}
Splitstep IMPL
\subsection{Finite Difference}
FD IMPL
\section{Results and Discussion}
RESULTS
\section{Conclusion}
CONCLUSION


\end{document}